\marginpar{\textcolor{red}{Vorlesung 8}}

\begin{deff}
	Ist $f \! : (\mathbb{R}^d,\cB(\mathbb{R}^d)) \rightarrow (\mathbb{R}^d,\cB(\mathbb{R}^{d'}))$ messbar, so heißt $f$ \textbf{Borel-messbar}.
\end{deff}

\begin{beispiel}\abs
	\begin{itemize}
		\item Jede stetige Abbildung $f \! : \mathbb{R} \rightarrow \mathbb{R}$  ist auch Borel-messbar. Warum?
		\begin{enumerate}[label=(\roman*)]
			\item Urbilder offener Mengen sind offen
			\item $\sigma(\{ O \! : O \subseteq \mathbb{R} \text{ offen} \}) = \mathcal B (\mathbb{R})$
			\item Satz \ref{S2}
		\end{enumerate}
	\item Indikatorfunktionen
	\begin{align*}
		\mathbf 1_A: \mathbb{R} \rightarrow \mathbb{R}, \:  \mathbf{1}_A(x) &= \begin{cases}
		1&: x\in A\\
		0&: x \notin A\\
		\end{cases}
	\end{align*}
	sind messbar genau dann wenn $A$ messbar ist weil wir alle Urbilder direkt hinschreiben k\"onnen:
	\begin{align*}
		f^{-1}(B) = \{ x\in \mathbb{R} \,:\, f(x) \in B \} &= \begin{cases}
		A&: 1 \in B, \: 0 \notin B\\
		A^C&: 1 \notin B, \: 0 \in B\\
		\mathbb{R}&: 1,0 \in B\\
		\emptyset&: 1,0 \notin B\\	
		\end{cases}.
	\end{align*}
	Das selbe funktioniert auch f\"ur Indikatorfunktionen $\mathbf{1}_A:\Omega\to\R$, das ist eine kleine \"Ubungensaufgabe.
	\end{itemize}
\end{beispiel}
Wie f\"ur stetige Abbildungen zeigt man, dass die Verkn\"upfung messbarer Abbildungen wieder messbar ist. Auch das ist eine kleine \"Ubungsaufgabe.
\begin{bem}
	Wir erinnern daran, dass
	\begin{align*}
		\cB(\mathbb{R}) &= \sigma(\{ (-\infty,t] \! : t \in \mathbb{R} \})= \sigma(\{ (-\infty,t) \! : t \in \mathbb{R} \})= \sigma(\{ (a,b) \! : a < b \}).
	\end{align*}
	% t ist hier gar nicht richtig definiert
	Wegen Satz \ref{S2} ist deshalb $f \! : \mathbb{R} \rightarrow \mathbb{R}$ Borel-messbar genau dann, wenn 
	\begin{gather*}
		f^{-1} ((-\infty,t]) = \{ x \! : f(x) \leq t \} =: \{ f \leq t \} \in \cB(\mathbb{R})
	\end{gather*}
	f\"ur alle $t\in\R$. Die Notation $\{f\leq t\}$ ist etwas ungewohnt, wird bei messbaren Abbildungen aber oft genutzt. Analog ist auch $f$ messbar genau dann, wenn
	\begin{align*}
		f^{-1} ((-\infty,t)) = \{ x \! : f(x) < t \} =: \{ f < t \} \in \cB(\mathbb{R})
	\end{align*}
	f\"ur alle $t\in\R$ oder 
	\begin{align*}
		f^{-1} ((a,b)) = \{ x \! : f(x) \in (a,b) \} =: \{ f \in (a,b) \}\in \cB(\mathbb{\R}).
	\end{align*}
\end{bem}

\begin{deff}\label{Kat}
	Sei $f \! : \Omega \rightarrow \Omega'$ für einen messbaren Raum $(\Omega', \cA')$. Dann ist \[ \cA := \big\{ f^{-1}(A') \! : A' \in \cA' \big\} \] eine $\sigma$-Algebra und $\mathcal A$ ist die kleinste $\sigma$-Algebra auf $\Omega$ f\"ur die $f$ $(\cA, \cA')$-messbar. Wir nennen die $\sigma$-Algebra $\cA$ auch $\sigma(f)$.
\end{deff}

\begin{proof}
	Übung.
\end{proof}
\begin{beispiel}\abs
	\begin{itemize}
		\item $\sigma(\mathbf{1}_A) = \{ \emptyset, \Omega, A, A^C \}$
		\item Sei $f \! : \mathbb{R} \rightarrow \mathbb{R}, \: f(x) = c$, dann ist $\sigma(f) = \{ \emptyset, \mathbb{R} \}$.
	\end{itemize}
\end{beispiel}

\begin{deff}
	Seien $(\Omega_i',\cA_i')$ messbare Räume, $f_i \! : \Omega \rightarrow \Omega_i'$ für $i \in I$. Dann ist \[\sigma(f_i, \: i \in I) := \sigma \Big(\bigcup_{i \in I}\sigma(f_i)\Big) = \sigma(\{ f_i^{-1}(A_i) \! : A_i \in \cA_i, \: i \in I \}) \] die kleinste $\sigma$-Algebra auf $\Omega$, bez\"uglich derer alle $f_i$ messbar sind.
\end{deff}

\section{Bildmaße oder \enquote{push-forward} eines Maßes}

Wir nutzten die Messbarkeit einer Abbildung $f:(\Omega, \mathcal A)\to (\Omega', \mathcal A')$, um ein Ma\ss{} $\mu$ auf $(\Omega, \mathcal A)$ auf ein Ma\ss{} $\mu_f$ auf $(\Omega', \mathcal A')$ r\"uberzuschieben (deshalb \enquote{push-forward}). In dem Stochastikteil werden wir noch sehen, dass der push-forward extrem wichtig ist.

\begin{satz}\label{pushf}
	Sei $f:(\Omega, \cA) \rightarrow (\Omega', \cA')$ messbar und $\mu$ ein Maß auf $(\Omega, \cA)$. Dann ist 
	\[ \mu_f(B) := \mu\left(f^{-1}(B)\right),\quad B\in \mathcal A', \] ein Maß auf $(\Omega', \cA')$. Dieses Maß heißt \enquote{Bildmaß} oder \enquote{push-forward}.
\end{satz}

\begin{proof}
	$\mu_f$ ist wohldefiniert weil $f$ messbar ist und daher $f^{-1}(B)\in \mathcal A$, wo $\mu$ definiert ist. Die Positivit\"at von $\mu_f$ folgt nat\"urlich direkt aus der Positivit\"at von $\mu$. Checken wir noch die zwei definierenden Eigenschaften eines Ma\ss es:
	\begin{itemize}
		\item[(i)] $\mu_f ( \emptyset) = \mu\left(f^{-1}(\emptyset)\right) = \mu(\emptyset) = 0$
		\item[(ii)] Seien $A_1,A_2,... \in \cA'$ paarweise disjunkt, dann folgt aus der Definition und den Ma\ss eigenschaften von $\mu$
		\begin{align*}
			\mu_f\Big(\bigcupdot_{n=1}^{\infty}A_n\Big)&\overset{\text{Def.}}{=} \mu\Big(f^{-1}\Big(\bigcupdot_{n=1}^{\infty}A_n\Big)\Big)\\
			& \overset{\text{Urbild}}{=} \mu\Big(\bigcupdot_{n=1}^{\infty}f^{-1}(A_n)\Big)\\
			&\overset{\mu\text{ Ma\ss}}{=} \sum\limits_{n=1}^{\infty} \mu\Big( f^{-1}(A_n)\Big) \overset{\text{Def.}}{=} \sum\limits_{n=1}^{\infty} \mu_f(A_n).\\
		\end{align*}
		Damit ist $\mu_f$ auch $\sigma$-additiv.
	\end{itemize}
\end{proof}

\begin{beispiel}
	Sei $f \! : \mathbb{R} \rightarrow \mathbb{R}$, $f(x) = x + a$. $f$ ist Borel-messbar weil $f$ stetig ist. Sei $\mu:=\lambda$ das Lebesgue-Maß auf $(\R, \mathcal B(\R))$, was ist dann der push-forward $\mu_f$? $\mu_f$ ist laut Satz \ref{pushf} ein Maß, aber welches?\smallskip
	
	\textit{Behauptung:} $\mu_f = \lambda$. Berechnen wir dazu $\mu_f$ auf einem $\cap$-stabilen Erzeuger von $\mathcal B(\R)$:
	\begin{gather*}
		\mu_f((c,b]) \overset{\text{Def.}}{=} \mu(f^{-1}\big(c,b\big]) = \lambda((c-a,b-a]) = (b-a) - (c-a) = b-c=\lambda((c,b]).
	\end{gather*}	
	Weil $\cE = \{ (a,b] \! : a,b \in \mathbb{R} \}$ $\cap$-stabil ist mit $\sigma(\cE) = \cB(\mathbb{R})$, gilt aufgrund von Folgerung \ref{folg} auch $\lambda = \mu_f$. In anderen Worten: Es gilt $$\lambda(B) = \lambda(B + x)$$ f\"ur alle $x \in \mathbb{R}$, wobei $B+x := \{ b+x:b\in B \}$. Man sagt, dass Lebesgue-Maß  \textbf{translationsinvariant}, Verschieben von Mengen \"andert das Lebesguema\ss{} nicht.
\end{beispiel}

\section{Messbare numerische Funktionen}

Wir nutzen wie in Kapitel 1 die erweiterte Zahlengerade $\overline \R=[-\infty,+\infty]$. Dabei nutzen wir die definierten \glqq Rechenregeln\grqq{} aus Kapitel 1 und auch die Konvergenzen am Rand $a_n \rightarrow +\infty$, $n \to \infty$, und $a_n \rightarrow -\infty$, $n \to \infty$, wie in Analysis 1 definiert. Oft schreiben wir $\infty$ statt $+\infty$. Auf $\overline \R$ definieren wir auch eine $\sigma$-Algebra:
\begin{align*}
	\mathcal B(\overline \R):=\big \{ B\subseteq \overline \R\,: \, B\cap \R\in \mathcal B(\R)\big \}.
\end{align*}
Kurz \"uberlegen zeigt uns, dass $\mathcal B(\overline \R)$ folgende Mengen enth\"alt: alle $B\in \mathcal B(\R)$, sowie $B \cup \{+\infty\}$, $B\cup \{-\infty\}$ und $B\cup \{-\infty,+\infty\}$.
\begin{deff}
	Sei $(\Omega, \cA)$ ein messbarer Raum, $f \! : \Omega \rightarrow \overline{\mathbb{R}}$ heißt \textbf{messbare numerische Funktion}, falls $f$ $(\mathcal A, \cB(\overline{\mathbb{R}}))$-messbar ist.
\end{deff}

\begin{bem}\abs
	\begin{enumerate}[label=(\roman*)]
		\item Jede Borel-messbare Funktion $f \! : \Omega \rightarrow \mathbb{R}$ ist auch eine messbare numerische Funktion, denn $f^{-1} (A \cup B) = f^{-1}(B) \in \cB(\mathbb{R}) $ für $A \in \{ \{+\infty\}, \{-\infty\}, \{+\infty,-\infty\} \}$.
		\item Aussagen für messbare reelle Abbildungen gelten ganz analog für messbare numerische Abbildungen. So gilt etwa: $f \! : \Omega \rightarrow \overline{\mathbb{R}}$ ist $(\cA, \cB(\mathbb{\overline R}))$ messbar genau dann, wenn $\{ f \leq t \} \in \cA$ f\"ur alle $t \in \overline{\mathbb{R}}$. Das folgt auch aus Satz \ref{S2} weil $\cE = \{ [-\infty,t] \! : t \in \overline{\mathbb{R}} \}$ die $\sigma$-Algebra $\cB(\mathbb{R})$ erzeugt.
	\end{enumerate}
\end{bem}

\begin{deff}
	Für $a,b \in \overline{\mathbb{R}}$ definieren wir
	\begin{align*}
		a \land b &:= \min\{ a,b \},\\
		a \lor b &:= \max\{ a,b \},\\
		a^{+} &:= \max\{ 0, a \},\\
		a^{-} &:= -\min\{ 0, a \}.
	\end{align*}
	F\"ur numerische Funktionen werden analog punktweise  $f \land g$, $f \lor g$, $f^{+}$, $f^{-}$ definiert. $f^{+}$ heißt \textbf{Positivteil} von f und $f^{-}$ \textbf{Negativteil} von f.
\end{deff}
Es gelten direkt aus der Definition folgende wichtige Identit\"aten
\begin{align*}
	f = f^{+} - f^{-}\quad \text{ und }\quad |f| = f^{+} + f^{-},
\end{align*}
die uns zeigen wieso es oft reicht $f^+$ und $f^-$ zu untersuchen.